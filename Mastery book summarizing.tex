\documentclass{article}
\usepackage{hyperref}
\title{Mastery book summarizing}
\begin{document}
\maketitle
	\section{Introduction}
	I want to read this book because in my current organization of work, i stumbled on the task of active practice of interviews when i added the problem of "maybe i am bad at interviews".
	This task is weird since i don't have a similar task. It's the only one that contains these components:
	\begin{itemize}
		\item repetition to be prepared when really needed
		\item separate reviewing from main all reviews task
		\item research of tips and feedback
	\end{itemize}
	It reminded me of skill based living, RCAT network, the book mastery. The introduction talked about concepts i think about all the time. When you finish you should write, a clean summary
	\section{Reading sessions summaries}
	\subsection{07/12/2025}
	\subsubsection{statements}
	We can experience life in passive mode (mediocrity) in which we only have dreams, desires and obsessive thoughts. Sometimes, we can experience life in an achieving mode: a crisis or a project. We have a feeling of needing to get something practical done. This book talks of how to enter and maintain a state of mastery because successful people use/follow mastery for every day tasks. Mastery is a process of 
	\begin{enumerate}
		\item apprenticeship: practice + gathering of rules. feelings of boredom, confusion, urgency must me managed to be able to move up the phases.
		\item creativity: challenging existing rules
		\item mastery: intuition: fast and creative decisions
	\end{enumerate}
	Humans created drugs, magic and religion seeking the sense of control and reality influence but the effortless path to practical power don't exist.
	People who reach excellence have one thing in common: the ability to practice harder which stems from their intensity of their desire to learn, which comes from the uniqueness of their project. This allows them to withstand the negative feelings of learning. Masters also feel the connections to skills deeper than other people. Intellectual power is not a direct cause for success and achievement but it is a factor.
	Society can push you to stop living by mastery principle when you get laughed at for trying to master a certain skill or being harassed or excluded for following an unusual path.
	\subsubsection{questions}
	\begin{itemize}
		\item is there anything i can use in my DAA tree?
		\item he started talking about life, why there is no mention of none career mastery
		\item how can I navigate the hierarchy of skills? I mean should i become the master of individual skills first? should i master skills decomposition first?
	\end{itemize}
	\subsubsection{Opinion}
	\begin{itemize}
		\item The introduction is so bad. There is no chain of thought. I think that i need to only read books that have a quality reference table only.
		\item The audiobook have problems.
		\item This book is oriented to professional mastery only.
		\item I don't have a research question.
	\end{itemize}
	
	
	 
	\subsubsection{Finish bookmark}
	\href{https://youtu.be/T5zHkmkBtR0?t=831}{831}
	\subsection{07/12}
	\subsubsection{statements}
	When you lose inner calling, your work becomes mechanical, therefore you live for immediate pleasure, both of these causes makes you increasingly live passively. You can become depressed without understanding that the root cause is the lack of creativity expression. The progression from one phase of mastery to another becomes a ritual process. Now he says that the phases are not linear because skills must be maintained by learning activities that acquired it.
	\subsubsection{questions}
	\begin{itemize}
		\item how can i insert a section representing the book's chapter inside today's reading?
		\item is it true that addictions indulgence's cause is lack of inner calling's passion? simplistic not?
	\end{itemize}
	\subsubsection{opinions}
	\begin{itemize}
		\item still no references
		\item still dispatched toughts
	\end{itemize}
	\subsubsection{bookmark}
	\href{https://youtu.be/T5zHkmkBtR0?t=1094}{1094}
	\subsection{08/12}
	\subsubsection{Statements}
	You should improve your understanding of like and dislikes of skills and acitivities to improve your inner calling.
	\begin{itemize}
		\item Leonardi di vinci from young age until death always followed distinct and different skills and professional projects from peers. He also like these fields so much that he spends time learning and practicing instead of wasting time or indulging.
		\item You can like activities more than other people. If you understand that you should pursue them, master them and figure out how to make a living out of them, you will reach mastery. Uniqueness research.
		\item Figuring out life's task comes in three stages. reconnect with inclinations uniqueness. 2, plan redirection of current career.3, start with something somewhat passionate, until passionate niche. in globalized economy difference is what matters.
		\item These are 5 strategies to deal with obstables of this action plan.
	\end{itemize}
	\subsubsection{questions}
	\begin{itemize}
		\item can i implement anything?
		\item Is there a finishable task i can do?
		\item Is there a recurrent task i can do
		\begin{itemize}
			\item i can make it recurrent to evaluaet how much i like certain activities and skills
			\item i can create a recurrent task of updating life mission?
		\end{itemize}
	\end{itemize}
	\subsubsection{opinion}
	\begin{itemize}
		\item i did not like: some people never become who they are (mask, really)
		\item still talking about the mastery of professional skills not life skills.
	\end{itemize}
	\subsubsection{bookmark}
	\href{https://youtu.be/T5zHkmkBtR0?t=2138}{2138}
\end{document}
