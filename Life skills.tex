\documentclass{article}
\title{Life skills : reminders and problems}
\begin{document}
\maketitle
\section{Introduction}
Active learning of life skills prevents creating and prolonging problems. A skill = repetition + knowledge + experiment.
- steps are
	- extract lessons from recent mistakes and solutions
	- read, note related statements and questions, answer them
	- Simplifie and memorize one note to apply knowledge faster.
- mistakes are
	- no separation of concerns 
	- impatience
	- execution escapism
	- experimenting based learning too early
Examples of life skills (maybe all?)
- Quitting addictions
- self caring
- Reducing spendings
- Getting a D
- Getting a job
- Building TSL
- Problem solving, preventing, analyzing, prioritizing
Reminder from ChatGpt
- improvement = execution x reviewing x researching

My question is how does researching (improvement) help
- preventing problems
- solving problems
	- faster
	- efficientlier
- analyzing problems
The answer (mine)
- Researching and memorizing reduces the effort of brainstorming which reduces DTSBF
- Prevents taking too much time in execution and ineffective execution
- prepares you for new problems (some people just know career or family, importance of not experimenting with drugs, )
- If you read about other people suffering and their regrets you might start making wise decisions and be more responsible.
- Makes analyze problems better (rcat, 5w) and understand priorities better
\section{Definitions}
apprendre: retenir dans sa mémore des connaissance ou acquérir une compétence
Connaissance: informations, idées, savoirs, notions, concepts sur un domaine
compétence: capacité reconnue dans un domaine / aptitude à accomplir une tache
information : ensemble de données sur un sujet
idée: pensée, representation mentale de qqch
savoir: avoir dans la mémoire; ensemble de connaissances sur un sujet
notion: connaissance élementaire
concept: idée abstraite
accomplir: achever entierement, réaliser compléteent
(réalisation, complétion, achevement, accomplissement ?)
tache, travail donnée à accomplir
donnée information brute servant pour un raisonnement/ analyse
pensée toute opération de l'intelligence, réflexion
travail, effeort soutenu pour un accomlmplissement
abstrait: non visible mais concevable mentalement
raisonnement enchainement d'arguments, utilisation de la raison
analyse procédé de raisonnement partant des parties au global
intellligence: développment de la faculté de compréhesion et de raisonnement
réflexion: analyse des idées , oncepts et situations
raison: prise de décisions basées sur la reflexion et la logique
concevoir ?
comprendre: saisir mentalement le sens et la signification
argument raisonnement aboutissant a une conséquence
logique raisonnement valide et rationnel ou vailidité et rationnalité desarguments
exercice : toute activité visant le maintient et l'amélioration des compétences.
\section{Types of tasks}
To solve most of your problems
- Maintenance tasks are recurrent to prevent reappearing problems.
- Waterfall tasks are to be finished once to change your situational problems.
- Learning tasks allow you to prevent creating and prolonging problems
- Reviewing allow you to solve problems
\section{questions}
I struggle in life because i lack these skills and their components:
\begin{itemize}
	\item self regulation
	\item executive functioning
	\item self management
	\item the actual life skills
\end{itemize}
\end{document}